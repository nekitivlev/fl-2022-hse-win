\documentclass[12pt]{article}
\usepackage[left=2cm,right=2cm,top=1cm,bottom=1cm,bindingoffset=0cm]{geometry}
\usepackage[utf8x]{inputenc}
\usepackage[english,russian]{babel}
\usepackage{cmap}
\usepackage{amssymb}
\usepackage{amsmath}
\usepackage{pifont}
\usepackage{tikz}
\usepackage{verbatim}
\usepackage{enumitem}
\usepackage{hyperref}

\pagenumbering{gobble}

\begin{document}

\begin{center}
{\LARGE Формальные языки}

{\Large Самостоятельная работа}

{\large 6 октября 2022}
\end{center}

\bigskip

\begin{center}
  \LARGE Порядок проведения самостоятельной работы
\end{center}

\begin{itemize}
  \item Это самостоятельная работа по первым темам по КС языкам. Она будет заменять маленькую домашку до понедельника. Исправлять после проверки ничего будет нельзя.
  \item Самостоятельная работа должна выполняться каждым индивидуально.
  \item Самостоятельную работу можно писать ручкой на листе бумаги. Если есть возможность отсканировать выполненную работу --- отсканируйте, иначе достаточно качественной фотографии.
  \item Перед решением каждого задания обязательно укажите номер задачи. Обязательно убедитесь, что решаете положенный вам вариант. Вариант будет один на все задачи контрольной.
  \item Вариант смотреть в таблице: \url{https://docs.google.com/spreadsheets/d/1Bf9yl5dwLmcDukRUSlCU4wT_Aasw17wtaWQ4Mcsmrrs/edit?usp=sharing}
  \item Дедлайн по самостоятельной --- 23:59 9 октября. Тем, кто пришлет выполненную самостоятельную за время пары --- бонусный балл.
  \item Любые соображения, которые привели вас к решению, целесообразно написать.
  \item Проверьте, что у грамматик явно указан стартовый нетерминал (один). Убедитесь, что если вас просят построить дерево вывода, вы строите дерево, и оно является деревом вывода. Убедитесь, что построенный вами левосторонний вывод является левосторонним, и что он является выводом.
  \item Прочитайте, что такое язык Дика (\url{https://bit.ly/2UH0hus}) и префиксная/постфиксная запись арифметических выражений (\url{https://bit.ly/3feP5ie}, \url{https://bit.ly/2UT9gcl}).
  \item Читайте задания внимательно.
\end{itemize}

\newpage


\begin{enumerate}
\setlength\itemsep{1em}

  \item Построить грамматику для языка:
  \begin{enumerate}[label=\arabic*)]
    \setlength\itemsep{0.8em}
    \item Язык Дика с двумя типами скобок \verb!(!, \verb!)! и \verb![!, \verb!]!, в котором открывающая круглая скобка \verb!(! не может встречаться сразу после открывающей квадратной скобки \verb![!.
    \item Язык Дика с двумя типами скобок \verb!(!, \verb!)! и \verb![!, \verb!]!, в котором закрывающая квадратная скобка \verb!]! не может встречаться сразу после закрывающей круглой скобки \verb!)!.
    \item Язык Дика с двумя типами скобок \verb!(!, \verb!)! и \verb![!, \verb!]!, в котором открывающая круглая скобка \verb!(! может встречаться только сразу после открывающей квадратной скобки~\verb![!.
    \item Язык Дика с двумя типами скобок \verb!(!, \verb!)! и \verb![!, \verb!]!, в котором закрывающая квадратная скобка \verb!]! может встречаться только перед закрывающей круглой скобкой \verb!)!.
    \item Язык Дика с двумя типами скобок \verb!(!, \verb!)! и \verb![!, \verb!]!, в котором открывающая скобка не может встречаться сразу после закрывающей скобки другого вида (\verb!(! не может быть после \verb!]!, а \verb![! не может быть после \verb!)!).
    \item Язык Дика с двумя типами скобок \verb!(!, \verb!)! и \verb![!, \verb!]!, в котором открывающая скобка не может встречаться сразу после закрывающей скобки такого же вида (\verb!(! не может быть после \verb!)!, а \verb![! не может быть после \verb!]!).
    \item Язык корректных арифметических выражений с операциями \verb!+! и \verb!*! над числами \verb!0! и \verb!1! в префиксной записи.
    \item Язык корректных арифметических выражений с операциями \verb!+! и \verb!*! над числами \verb!0! и \verb!1! в постфиксной записи.
    \item $\{a^{3n} b^m \mid 1 \leq n \leq m \leq 2n \}$.
    \item $\{a^{2n} b^m \mid 1 \leq 3n \leq m \leq 4n \}$.
  \end{enumerate}

  \item Привести левостронний вывод для трех самых коротких цепочек из языка. Привести две цепочки произвольной длины, которые не принадлежат языку.

  \item Проверить, является ли построенная грамматика LL(1). Если является, привести таблицу анализатора и продемонстрировать успешный и неуспешный синтаксический анализ на 2 цепочках длины не меньше 7, для корректной строки построить дерево вывода. Если нет, обосновать.

  \item Можно ли проанализировать такой язык при помощи алгоритма CYK? Если можно, привести таблицу анализатора и продемонстрировать успешный и неуспешный синтаксический анализ на 2 цепочках длины не меньше 7, для корректной строки построить дерево вывода. Если нет, обосновать.

\end{enumerate}

\end{document}
